\problemset{Формальные языки}


\begin{problem}
  Для грамматики скобочных последовательностей S $\rightarrow$ (S)| SS | $\varepsilon$:
  \begin{enumerate}[(a)]
    \item Построить LR(0) автомат и LR(0) таблицу.
    \item Если не удалось, построить SLR(1) таблицу для той же грамматики.
    \item Если не удалось, построить CLR(1) автомат и таблицу для той же грамматики.
    \item Если не удалось, подумать и написать, почему так вышло.
    \item Если какую-нибудь таблицу построить все-таки удалось, промоделировать с ней
    разбор строк ()(()) и ((): предоставить историю изменения стека и дерево разбора.
  \end{enumerate}
\end{problem}

Для данной грамматики невозможно построить $LR(k)$ автомат ни для 
какого $k$, так как она неоднозначна. Например, вот два 
различных дерева вывода для строки ``()'':

\begin{forest}
  [S
    [[(] 
      [S
        [{$\varepsilon$}]
      ]
     [)] 
    ]
  ]
\end{forest}

и

\begin{forest}
  [S
    [S
      [(]
      [S [{$\varepsilon$}] ]
      [)]
    ]
    [S [{$\varepsilon$}] ]
  ]
\end{forest}

\break

\begin{problem}
  Для языка арифметических выражений с вычитанием (-), делением (/) и скобками
  над алфавитом $\{0, 1\}$ с правильным приоритетом операций и ассоциативностью:
  \begin{enumerate}[(a)]
    \item Привести грамматику, подходящую для анализа алгоритмом из семейства LR.
    \item Построить LR(0) автомат и LR(0) таблицу.
    \item Если не удалось, построить SLR(1) таблицу для той же грамматики.
    \item Если не удалось, построить CLR(1) автомат и таблицу для той же грамматики.
    \item Промоделировать с ней разбор строк 0-1/(0-1) и 1//0: предоставить историю
    изменения стека и дерево разбора.
  \end{enumerate}
  Автомат можно не рисовать в виде графа: достаточно указать, из каких LR-item-ов
  состоят состояния, и предоставить таблицу переходов между состояниями.
\end{problem}

Грамматика:
\begin{lstlisting}
  S` -> S
  S -> S - A 
  S -> A
  A -> A / N 
  A -> N
  N -> 0 
  N -> 1 
  N -> ( S )
\end{lstlisting}

Построить $LR(0)$ не удалось, но зато удалось построить 
$SLR(1)$.

Множества $FOLLOW$:
\begin{align*}
  FOLLOW(S`) &= \{ \$ \} \\
  FOLLOW(S) &= \{ -, ), \$ \} \\
  FOLLOW(A) &= \{ /, -, ), \$ \} \\
  FOLLOW(N) &= \{ /, -, ), \$ \}
\end{align*}  

Список состояний автомата:

1.
\begin{lstlisting}
  S` -> . S
  S -> . S - A
  S -> . A
  A -> . A / N
  A -> . N
  N -> . 0
  N -> . 1
  N -> . ( S )
\end{lstlisting}

2.
\begin{lstlisting}
  N -> ( . S )
  S -> . S - A
  S -> . A
  S -> . A / N
  A -> . N
  N -> . 0
  N -> . 1
  N -> . ( S )
\end{lstlisting}
\break
3.
\begin{lstlisting}
  S` -> S .
  S -> S . - A
\end{lstlisting}

4.
\begin{lstlisting}
  S -> S - . A
  A -> . A / N
  A -> . N
  N -> . 0
  N -> . 1
  N -> . ( S )
\end{lstlisting}

5.
\begin{lstlisting}
  N -> 0 .  
\end{lstlisting}

6.
\begin{lstlisting}
  N -> 1 .
\end{lstlisting}

7.
\begin{lstlisting}
  S -> A .
  A -> A . / N
\end{lstlisting}

8.
\begin{lstlisting}
  A -> N .
\end{lstlisting}

9.
\begin{lstlisting}
  A -> A / . N
  N -> . 0
  N -> . 1
  N -> . ( S )  
\end{lstlisting}

10.
\begin{lstlisting}
  A -> A / N .  
\end{lstlisting}

11.
\begin{lstlisting}
  N -> ( S ) .
\end{lstlisting}
\break
12.
\begin{lstlisting}
  N -> ( S . )
  S -> S . - A
\end{lstlisting}

13.
\begin{lstlisting}
  S -> S - A .
  A -> A . / N
\end{lstlisting}

Таблица разбора:
\begin{center}
  \begin{tabular}{ l || c | c | c | c | c | c | c || c | c | c }
      & /     & -     & (     & )     & 0     & 1     & \$    & S     & A     & N \\ \hline  
    1 &       &       & $s_2$ &       & $s_5$ & $s_6$ &       & 3     & 7     & 8 \\ 
    2 &       &       & $s_2$ &       & $s_5$ & $s_6$ &       & 12    & 7     & 8 \\
    3 &       & $s_4$ &       &       &       &       & acc   &       &       &   \\
    4 &       &       & $s_2$ &       & $s_5$ & $s_6$ &       &       & 13    & 8 \\
    5 & $r_6$ & $r_6$ &       & $r_6$ &       &       & $r_6$ &       &       &   \\
    6 & $r_7$ & $r_7$ &       & $r_7$ &       &       & $r_7$ &       &       &   \\
    7 & $s_9$ & $r_3$ &       & $r_3$ &       &       & $r_3$ &       &       &   \\
    8 & $r_5$ & $r_5$ &       & $r_5$ &       &       & $r_5$ &       &       &   \\
    9 &       &       & $s_2$ &       & $s_5$ & $s_6$ &       &       &       & 10\\
   10 & $r_4$ & $r_4$ &       & $r_4$ &       &       & $r_4$ &       &       &   \\
   11 & $r_8$ & $r_8$ &       & $r_8$ &       &       & $r_8$ &       &       &   \\
   12 &       & $s_4$ &       & $s_{11}$&       &       &       &       &       &   \\
   13 & $s_9$ & $r_2$ &       & $r_2$ &       &       & $r_2$ &       &       &   
  \end{tabular}  
\end{center}

Разбор некорректной строки ``1 // 0''
\begin{center}
  \begin{tabular}{ l | c }
       STACK & STREAM  \\ \hline  
       $\oo{1}$   & 1 / / 0 \$ \\
       $\oo{1}$ 1 $\oo{6}$  & / / 0 \$ \\
       $\oo{1}$ N $\oo{8}$  & / / 0 \$ \\
       $\oo{1}$ A $\oo{7}$  & / / 0 \$ \\
       $\oo{1}$ A $\oo{7}$ / $\oo{9}$  & / 0 \$ 
  \end{tabular}  
\end{center}
Ошибка. Ячейка $(9, /)$ в таблице пуста.

\break

Разбор корректной строки ``0 - 1 / ( 0 - 1 )''
\begin{center}
  \begin{tabular}{ l | c }
       STACK & STREAM  \\ \hline  
       $\oo{1}$   & 0 - 1 / ( 0 - 1 ) \$ \\
       $\oo{1}$ 0 $\oo{5}$   & - 1 / ( 0 - 1 ) \$ \\
       $\oo{1}$ N $\oo{8}$   & - 1 / ( 0 - 1 ) \$ \\
       $\oo{1}$ A $\oo{7}$   & - 1 / ( 0 - 1 ) \$ \\
       $\oo{1}$ S $\oo{3}$   & - 1 / ( 0 - 1 ) \$ \\
       $\oo{1}$ S $\oo{3}$ - $\oo{4}$  & 1 / ( 0 - 1 ) \$ \\
       $\oo{1}$ S $\oo{3}$ - $\oo{4}$ 1 $\oo{6}$ & / ( 0 - 1 ) \$ \\
       $\oo{1}$ S $\oo{3}$ - $\oo{4}$ N $\oo{8}$ & / ( 0 - 1 ) \$ \\
       $\oo{1}$ S $\oo{3}$ - $\oo{4}$ A $\oo{13}$ & / ( 0 - 1 ) \$ \\
       $\oo{1}$ S $\oo{3}$ - $\oo{4}$ A $\oo{13}$ / $\oo{9}$ & ( 0 - 1 ) \$ \\
       $\oo{1}$ S $\oo{3}$ - $\oo{4}$ A $\oo{13}$ / $\oo{9}$ ( $\oo{2}$ & 0 - 1 ) \$ \\
       $\oo{1}$ S $\oo{3}$ - $\oo{4}$ A $\oo{13}$ / $\oo{9}$ ( $\oo{2}$ 0 $\oo{5}$ & - 1 ) \$ \\
       $\oo{1}$ S $\oo{3}$ - $\oo{4}$ A $\oo{13}$ / $\oo{9}$ ( $\oo{2}$ N $\oo{8}$ & - 1 ) \$ \\
       $\oo{1}$ S $\oo{3}$ - $\oo{4}$ A $\oo{13}$ / $\oo{9}$ ( $\oo{2}$ A $\oo{7}$ & - 1 ) \$ \\
       $\oo{1}$ S $\oo{3}$ - $\oo{4}$ A $\oo{13}$ / $\oo{9}$ ( $\oo{2}$ S $\oo{12}$ & - 1 ) \$ \\
       $\oo{1}$ S $\oo{3}$ - $\oo{4}$ A $\oo{13}$ / $\oo{9}$ ( $\oo{2}$ S $\oo{12}$ - $\oo{4}$ & 1 ) \$ \\
       $\oo{1}$ S $\oo{3}$ - $\oo{4}$ A $\oo{13}$ / $\oo{9}$ ( $\oo{2}$ S $\oo{12}$ - $\oo{4}$ 1 $\oo{6}$ & ) \$ \\
       $\oo{1}$ S $\oo{3}$ - $\oo{4}$ A $\oo{13}$ / $\oo{9}$ ( $\oo{2}$ S $\oo{12}$ - $\oo{4}$ A $\oo{13}$ & ) \$ \\
       $\oo{1}$ S $\oo{3}$ - $\oo{4}$ A $\oo{13}$ / $\oo{9}$ ( $\oo{2}$ S $\oo{12}$  & ) \$ \\
       $\oo{1}$ S $\oo{3}$ - $\oo{4}$ A $\oo{13}$ / $\oo{9}$ ( $\oo{2}$ S $\oo{12}$ ) $\oo{11}$ & \$ \\
       $\oo{1}$ S $\oo{3}$ - $\oo{4}$ A $\oo{13}$ / $\oo{9}$ N $\oo{10}$  & \$ \\
       $\oo{1}$ S $\oo{3}$ - $\oo{4}$ A $\oo{13}$ & \$ \\
       $\oo{1}$ S $\oo{3}$ & \$ \\
       acc &
  \end{tabular}  
\end{center}

Дерево разбора.
\begin{forest}
  [S`
    [S
      [S
        [A
          [N
            [0]
          ]
        ]
      ]
      [-]
      [A
        [A
          [N
            [1]
          ]
        ]
        [/]
        [N
          [(]
          [S
            [S
              [A
                [N
                  [0]
                ]
              ]
            ]
            [-]
            [A
              [N
                [1]
              ]
            ]
          ]
          [)]
        ]
      ]
    ]
  ]
\end{forest}