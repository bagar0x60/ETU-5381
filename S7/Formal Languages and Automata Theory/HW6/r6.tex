\problemset{Формальные языки}

\graphicspath{{../1/}{../2/}}


\begin{problem}
  Привести в Нормальную Форму Хомского грамматику списков (грамматика ниже).
\end{problem}

\begin{itemize}  
  \item Добавьте в отчет промежуточные результаты преобразования на каждом шаге.
\end{itemize}

Исходная грамматика
\begin{lstlisting}
  S -> L ; S | L
  L -> a | [ S ]
\end{lstlisting}

Шаг 1 - Удаление стартового нетерминала из правых частей правил
\begin{lstlisting}
  S -> L ; S | L
  L -> a | [ S ]
  S3 -> S
\end{lstlisting}

Шаг 2 - Избавление от неодиночных терминалов
\begin{lstlisting}
  S -> L S4 S | L
  L -> a | S5 S S6
  S3 -> S
  S4 -> ;
  S5 -> [
  S6 -> ]
\end{lstlisting}

Шаг 3 - Удаление длинных правил
\begin{lstlisting}
  S -> S7 S | L
  L -> a | S8 S6
  S3 -> S
  S4 -> ;
  S5 -> [
  S6 -> ]
  S7 -> L S4
  S8 -> S5 S
\end{lstlisting}

Шаг 4 - Удаление эпсилон правил
\begin{lstlisting}
  S -> S7 S | L
  L -> a | S8 S6
  S3 -> S
  S4 -> ;
  S5 -> [
  S6 -> ]
  S7 -> L S4
  S8 -> S5 S
\end{lstlisting}

Шаг 5 - Удаление цепных правил
\begin{lstlisting}
  S -> S7 S | a | S8 S6
  L -> a | S8 S6
  S3 -> S7 S | a | S8 S6
  S4 -> ;
  S5 -> [
  S6 -> ]
  S7 -> L S4
  S8 -> S5 S
\end{lstlisting}


\begin{problem}
  Осуществить синтаксический анализ алгоритмом CYK для 2 списков, содержащих не меньше 7 терминалов: один список должно быть корректным, другой — нет. Для корректного 
  списка приведите дерево вывода. Используйте грамматику, полученную в прошлом задании.
\end{problem}
\begin{itemize}  
  \item Добавьте в отчет таблицы, которые строит алгоритм CYK. Не забывайте делать вывод
  о выводимости цепочки.
\end{itemize}

1. CYK таблица неверной строки ``[ [ a ] ; [ ] ]''

\csvautotabular{../2/grammar_1_table.csv}

Правая верхняя ячейка таблицы пуста, 
значит строка не выводима в данной грамматике.


2. CYK таблица строки ``[ [ a ; a ] ; a ]''

\csvautotabular{../2/grammar_2_table.csv}

Правая верхняя ячейка содержит начальный нетерминал S3, 
значит строка выводима.

Дерево разбора для этой строчки
\begin{forest}
  [S3
    [S8
      [S5 [{$[$}]]
      [S
        [S7
          [L
            [S8
              [S5 [{$[$}]]
              [S
                [S7
                  [L [a]]
                  [S4 [;]]
                ]
                [S [a]]
              ]
            ]
            [S6 [{$]$}]]
          ]
          [S4 [;]]
        ]
        [S [a]]
      ]
    ]
    [S6 [{$]$}]]
  ]
\end{forest}

\begin{problem}
  Построить LL(1) таблицу по грамматике списков.
\end{problem}

LL(1) грамматика списков
\begin{lstlisting}
  S -> L S'
  S' -> ; S | _eps
  L -> a | [ S ]
\end{lstlisting}

\break

LL(1) таблица для этой грамматики
\begin{center}
  \begin{tabular}{ l || c | c || c | c | c | c | c }
    N & FIRST                   & FOLLOW          & ; & a   & [   & ]             &  \$              \\ \hline  
    S & \{ a, [ \}              & \{ \$, ] \}     &   & LS' & LS' &               &                  \\ 
    S'& \{ ;, $\varepsilon$ \}  & \{ \$, ] \}     &; S&     &     & $\varepsilon$ & $\varepsilon$    \\ 
    L & \{ a, [ \}              & \{  ;, \$, ] \} &   & a   & [S] &               &      
  \end{tabular}  
\end{center}

\begin{problem}
Осуществить синтаксический анализ алгоритмом LL(1) для 
2 списков, содержащих не меньше 7 терминалов:
 один список должно быть корректным, другой — нет. 
 Для корректного списка приведите дерево вывода.
Используйте грамматику, полученную в прошлом задании.
\end{problem}
\begin{itemize}  
  \item Добавьте в отчет “историю” стека — то, в каком порядке в стек помещались символы
  грамматики (как мы делали на паре).
\end{itemize}

Разбор для неверной строки ``[ [ ] ; [ a ] ]''

\csvautotabular[respect dollar]{../4/incorrect.csv}
\break


Разбор для строки ``[ [ a ; a ] ; a ]''

\csvautotabular[respect dollar]{../4/correct.csv}
\break

Дерево вывода для этой строки\\
\begin{forest}
  [S
    [L
      [{$[$}]
      [S
        [L
          [{$[$}]
          [S
            [L [a]]
            [S'
              [;]
              [S
                [L [a]]
                [S' [$\varepsilon$]]
              ]
            ]
          ]
          [{$]$}]
        ]
        [S'
          [;]
          [S
            [L [a]]
            [S' [{$\varepsilon$}]]
          ]
        ]
      ]
      [{$]$}]
    ]
    [S'
      [{$\varepsilon$}]
    ]
  ]
\end{forest}