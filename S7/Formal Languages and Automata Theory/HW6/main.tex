\documentclass{amsart}
    \usepackage{ifxetex}
    \ifxetex
      \usepackage{fontspec}
      \usepackage{xunicode}
      \usepackage{xltxtra}
      \usepackage{xecyr}
      \setmainfont[Mapping=tex-text,Ligatures=TeX]{CMU Serif}
      \usepackage{polyglossia}
      \setdefaultlanguage{russian}
    \else
      \usepackage[utf8]{inputenc}
      \usepackage[T2A]{fontenc}
      \usepackage[english,russian]{babel}
      \usepackage{concrete}
    \fi
    \usepackage{amsthm,amsmath,amsfonts,amssymb,mathtools}
    \usepackage{fullpage}
    \usepackage{eufrak}
    \usepackage{listings}
    \usepackage{color}
    \usepackage{xcolor}


    \usepackage{graphicx}
    \usepackage{subcaption}

    \usepackage{hyperref}

    \usepackage{listings}

    \usepackage{csvsimple}
    \usepackage[linguistics]{forest}

    \lstset{
      basicstyle=\itshape,
      xleftmargin=3em,
      literate={->}{$\rightarrow$}{2}
              {_eps}{$\varepsilon$}{1}     
    }
    
    
    \newtheorem{problem}{Задача}
    
    \begin{document}
    
      \definecolor{dkgreen}{rgb}{0,0.6,0}
      \definecolor{gray}{rgb}{0.5,0.5,0.5}
      \definecolor{mauve}{rgb}{0.58,0,0.82}  
    
      \newcommand{\problemset}[1]{
        
        \begin{center}
          \Large #1
        \end{center}
      }
    
      \lstset{ %
        language=C++,                % the language of the code
        basicstyle=\footnotesize,           % the size of the fonts that are used for the code
        numbers=left,                   % where to put the line-numbers
        numberstyle=\tiny\color{gray},  % the style that is used for the line-numbers
        stepnumber=1,                   % the step between two line-numbers. If it's 1, each line 
                                        % will be numbered
        numbersep=5pt,                  % how far the line-numbers are from the code
        backgroundcolor=\color{white},      % choose the background color. You must add \usepackage{color}
        showspaces=false,               % show spaces adding particular underscores
        showstringspaces=false,         % underline spaces within strings
        showtabs=true,                 % show tabs within strings adding particular underscores
        frame=single,                   % adds a frame around the code
        rulecolor=\color{black!10},        % if not set, the frame-color may be changed on line-breaks within not-black text (e.g. comments (green here))
        tabsize=2,                      % sets default tabsize to 2 spaces
        captionpos=b,                   % sets the caption-position to bottom
        breaklines=true,                % sets automatic line breaking
        breakatwhitespace=false,        % sets if automatic breaks should only happen at whitespace
        title=\lstname,                   % show the filename of files included with \lstinputlisting;
                                        % also try caption instead of title
        keywordstyle=\color{blue},          % keyword style
        commentstyle=\color{dkgreen},       % comment style
        stringstyle=\color{mauve},        % string literal style
        escapeinside={\%*}{*)},            % if you want to add LaTeX within your code
        morekeywords={done, to},              % if you want to add more keywords to the set
      %  deletekeywords={...}              % if you want to delete keywords from the given language
      }
    
      \begin{tabbing}
    \hspace{11cm} \= Студент: \= Кобылянский Алексей \\
      \> Группа: \> 5381 \\
      \> Дата: \> \today
    \end{tabbing}
    \hrule
    \vspace{1cm}
    
    
      \problemset{Формальные языки}

\graphicspath{{../1/}{../2/}}


\begin{problem}
  Привести в Нормальную Форму Хомского грамматику списков (грамматика ниже).
\end{problem}

\begin{itemize}  
  \item Добавьте в отчет промежуточные результаты преобразования на каждом шаге.
\end{itemize}

Исходная грамматика
\begin{lstlisting}
  S -> L ; S | L
  L -> a | [ S ]
\end{lstlisting}

Шаг 1 - Удаление стартового нетерминала из правых частей правил
\begin{lstlisting}
  S -> L ; S | L
  L -> a | [ S ]
  S3 -> S
\end{lstlisting}

Шаг 2 - Избавление от неодиночных терминалов
\begin{lstlisting}
  S -> L S4 S | L
  L -> a | S5 S S6
  S3 -> S
  S4 -> ;
  S5 -> [
  S6 -> ]
\end{lstlisting}

Шаг 3 - Удаление длинных правил
\begin{lstlisting}
  S -> S7 S | L
  L -> a | S8 S6
  S3 -> S
  S4 -> ;
  S5 -> [
  S6 -> ]
  S7 -> L S4
  S8 -> S5 S
\end{lstlisting}

Шаг 4 - Удаление эпсилон правил
\begin{lstlisting}
  S -> S7 S | L
  L -> a | S8 S6
  S3 -> S
  S4 -> ;
  S5 -> [
  S6 -> ]
  S7 -> L S4
  S8 -> S5 S
\end{lstlisting}

Шаг 5 - Удаление цепных правил
\begin{lstlisting}
  S -> S7 S | a | S8 S6
  L -> a | S8 S6
  S3 -> S7 S | a | S8 S6
  S4 -> ;
  S5 -> [
  S6 -> ]
  S7 -> L S4
  S8 -> S5 S
\end{lstlisting}


\begin{problem}
  Осуществить синтаксический анализ алгоритмом CYK для 2 списков, содержащих не меньше 7 терминалов: один список должно быть корректным, другой — нет. Для корректного 
  списка приведите дерево вывода. Используйте грамматику, полученную в прошлом задании.
\end{problem}
\begin{itemize}  
  \item Добавьте в отчет таблицы, которые строит алгоритм CYK. Не забывайте делать вывод
  о выводимости цепочки.
\end{itemize}

1. CYK таблица неверной строки ``[ [ a ] ; [ ] ]''

\csvautotabular{../2/grammar_1_table.csv}

Правая верхняя ячейка таблицы пуста, 
значит строка не выводима в данной грамматике.


2. CYK таблица строки ``[ [ a ; a ] ; a ]''

\csvautotabular{../2/grammar_2_table.csv}

Правая верхняя ячейка содержит начальный нетерминал S3, 
значит строка выводима.

Дерево разбора для этой строчки
\begin{forest}
  [S3
    [S8
      [S5 [{$[$}]]
      [S
        [S7
          [L
            [S8
              [S5 [{$[$}]]
              [S
                [S7
                  [L [a]]
                  [S4 [;]]
                ]
                [S [a]]
              ]
            ]
            [S6 [{$]$}]]
          ]
          [S4 [;]]
        ]
        [S [a]]
      ]
    ]
    [S6 [{$]$}]]
  ]
\end{forest}

\begin{problem}
  Построить LL(1) таблицу по грамматике списков.
\end{problem}

LL(1) грамматика списков
\begin{lstlisting}
  S -> L S'
  S' -> ; S | _eps
  L -> a | [ S ]
\end{lstlisting}

\break

LL(1) таблица для этой грамматики
\begin{center}
  \begin{tabular}{ l || c | c || c | c | c | c | c }
    N & FIRST                   & FOLLOW          & ; & a   & [   & ]             &  \$              \\ \hline  
    S & \{ a, [ \}              & \{ \$, ] \}     &   & LS' & LS' &               &                  \\ 
    S'& \{ ;, $\varepsilon$ \}  & \{ \$, ] \}     &; S&     &     & $\varepsilon$ & $\varepsilon$    \\ 
    L & \{ a, [ \}              & \{  ;, \$, ] \} &   & a   & [S] &               &      
  \end{tabular}  
\end{center}

\begin{problem}
Осуществить синтаксический анализ алгоритмом LL(1) для 
2 списков, содержащих не меньше 7 терминалов:
 один список должно быть корректным, другой — нет. 
 Для корректного списка приведите дерево вывода.
Используйте грамматику, полученную в прошлом задании.
\end{problem}
\begin{itemize}  
  \item Добавьте в отчет “историю” стека — то, в каком порядке в стек помещались символы
  грамматики (как мы делали на паре).
\end{itemize}

Разбор для неверной строки ``[ [ ] ; [ a ] ]''

\csvautotabular[respect dollar]{../4/incorrect.csv}
\break


Разбор для строки ``[ [ a ; a ] ; a ]''

\csvautotabular[respect dollar]{../4/correct.csv}
\break

Дерево вывода для этой строки\\
\begin{forest}
  [S
    [L
      [{$[$}]
      [S
        [L
          [{$[$}]
          [S
            [L [a]]
            [S'
              [;]
              [S
                [L [a]]
                [S' [$\varepsilon$]]
              ]
            ]
          ]
          [{$]$}]
        ]
        [S'
          [;]
          [S
            [L [a]]
            [S' [{$\varepsilon$}]]
          ]
        ]
      ]
      [{$]$}]
    ]
    [S'
      [{$\varepsilon$}]
    ]
  ]
\end{forest}
    
    \end{document}
    