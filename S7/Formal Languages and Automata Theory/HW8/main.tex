\documentclass{amsart}
    \usepackage{ifxetex}
    \ifxetex
      \usepackage{fontspec}
      \usepackage{xunicode}
      \usepackage{xltxtra}
      \usepackage{xecyr}
      \setmainfont[Mapping=tex-text,Ligatures=TeX]{CMU Serif}
      \usepackage{polyglossia}
      \setdefaultlanguage{russian}
    \else
      \usepackage[utf8]{inputenc}
      \usepackage[T2A]{fontenc}
      \usepackage[english,russian]{babel}
      \usepackage{concrete}
    \fi
    \usepackage{amsthm,amsmath,amsfonts,amssymb,mathtools}
    \usepackage{fullpage}
    \usepackage{eufrak}
    \usepackage{listings}
    \usepackage{color}
    \usepackage{xcolor}


    \usepackage{graphicx}
    \usepackage{subcaption}

    \usepackage{hyperref}

    \usepackage{listings}

    \usepackage{csvsimple}
    \usepackage[linguistics]{forest}
    \usepackage[shortlabels]{enumitem}

    \usepackage{tikz}
    \newcommand*\oo[1]{\tikz[baseline=(char.base)]{
        \node[shape=circle,draw,inner sep=2pt,minimum size=0.6cm] (char) {#1};}}


    \lstset{
      basicstyle=\itshape,
      xleftmargin=3em,
      literate={->}{$\rightarrow$}{2}
              {_eps}{$\varepsilon$}{1} 
              {.}{$\bullet$}{1}  
    }
    
    
    \newtheorem{problem}{Задача}
    
    \begin{document}
    
      \definecolor{dkgreen}{rgb}{0,0.6,0}
      \definecolor{gray}{rgb}{0.5,0.5,0.5}
      \definecolor{mauve}{rgb}{0.58,0,0.82}  
    
      \newcommand{\problemset}[1]{
        
        \begin{center}
          \Large #1
        \end{center}
      }
    
      \lstset{ %
        language=C++,                % the language of the code
        basicstyle=\footnotesize,           % the size of the fonts that are used for the code
        numbers=left,                   % where to put the line-numbers
        numberstyle=\tiny\color{gray},  % the style that is used for the line-numbers
        stepnumber=1,                   % the step between two line-numbers. If it's 1, each line 
                                        % will be numbered
        numbersep=5pt,                  % how far the line-numbers are from the code
        backgroundcolor=\color{white},      % choose the background color. You must add \usepackage{color}
        showspaces=false,               % show spaces adding particular underscores
        showstringspaces=false,         % underline spaces within strings
        showtabs=true,                 % show tabs within strings adding particular underscores
        frame=single,                   % adds a frame around the code
        rulecolor=\color{black!10},        % if not set, the frame-color may be changed on line-breaks within not-black text (e.g. comments (green here))
        tabsize=2,                      % sets default tabsize to 2 spaces
        captionpos=b,                   % sets the caption-position to bottom
        breaklines=true,                % sets automatic line breaking
        breakatwhitespace=false,        % sets if automatic breaks should only happen at whitespace
        title=\lstname,                   % show the filename of files included with \lstinputlisting;
                                        % also try caption instead of title
        keywordstyle=\color{blue},          % keyword style
        commentstyle=\color{dkgreen},       % comment style
        stringstyle=\color{mauve},        % string literal style
        escapeinside={\%*}{*)},            % if you want to add LaTeX within your code
        morekeywords={done, to},              % if you want to add more keywords to the set
      %  deletekeywords={...}              % if you want to delete keywords from the given language
      }
    
      \begin{tabbing}
    \hspace{11cm} \= Студент: \= Кобылянский Алексей \\
      \> Группа: \> 5381 \\
      \> Дата: \> \today
    \end{tabbing}
    \hrule
    \vspace{1cm}
    
    
      \problemset{Формальные языки}
\graphicspath{{img/}}

\begin{problem}
  Построить магазинные автоматы, распознающие следующие языки:
\end{problem}

$\bullet \: \{ a^nb^{m+n}c^m \mid n,m \geq 1 \}$
\begin{figure}[h]
  \includegraphics[width=0.6\linewidth]{sm1.png}
  \caption{Автомат 1}
\end{figure}

$\bullet \: \{ a^nb^mc^n \mid n,m \geq 1 \}$
\begin{figure}[h]
  \includegraphics[width=0.6\linewidth]{sm2.png}
  \caption{Автомат 2}
\end{figure}

$\bullet \: \{ a^nb^{2n} \mid n \geq 1 \}$
\begin{figure}[h]
  \includegraphics[width=0.6\linewidth]{sm3.png}
  \caption{Автомат 3}
\end{figure}

\break

\begin{problem}
  Доказать или опровергнуть, что следующие языки
  являются контекстно-свободными:
\end{problem}

$\bullet \: \{ a^{n+m}b^mc^nd^m \mid n, m \geq 0 \}$

Для всех слов $s$ данного языка должно выполняться
\begin{align*}
  |s|_a &= |s|_c + |s|_b \\
  |s|_b &= |s|_d \\
\end{align*}

Предположим, что язык регулярный, тогда должна выполняться лемма о накачке.
$p$ - конста из леммы.
Возьмем слово $s$ принадлежащее нашему языку $s = a^{2p}b^pc^pd^p = uvwxy, |vwx| \leq p, |vx| \geq 1$. 
Рассмотрим слово $s' = uv^2wx^2y$.

1. $vwx = a^i, i \leq p$ тогда
\begin{align*}
  |s'|_a &= |s|_a + \Delta, \Delta \geq 1 \\
  |s'|_c + |s'|_b &= |s|_c + |s|_b = |s|_a \neq |s'|_a
\end{align*}

2. $vwx = a^ib^j, i + j \leq p$ тогда
\begin{align*}
  |s'|_a + |s'|_b &= |s|_a + |s|_b + \Delta, \Delta \geq 1 \\
  |s'|_c + 2|s'|_d &= |s|_c + 2|s|_d = |s|_a + |s|_b \neq |s'|_a + |s'|_b
\end{align*}

3. $vwx = b^i, i \leq p$ рассуждения идентичны рассуждениям в пункте 2

4. $vwx = b^ic^j, i + j \leq p$ тогда
\begin{align*}
  |s'|_b + |s'|_c &= |s|_b + |s|_c + \Delta, \Delta \geq 1 \\
  |s'|_a &= |s|_a = |s|_b + |s|_c \neq |s'|_b + |s'|_c
\end{align*}

5. $vwx = c^i, i \leq p$ рассуждения идентичны рассуждениям в пункте 4

6. $vwx = c^id^j, i + j \leq p$ тогда
\begin{align*}
  |s'|_c + |s'|_d &= |s|_c + |s|_d + \Delta, \Delta \geq 1 \\
  |s'|_a &= |s|_a = |s|_c + |s|_d \neq |s'|_c + |s'|_d
\end{align*}

7. $vwx = d^i, i \leq p$ рассуждения идентичны рассуждениям в пункте 6

Из данных рассуждений следует, что $s'$ не может принадлежать языку,
значит лемма о накачке не выполняется и язык не регулярный.\\

$\bullet \: \{ a^nb^na^n \mid n \geq 0 \} \cup 
  \{a^kb^lc^m \mid k \geq 0, l \geq 0, m \geq 20 \}$

Заметим, что
\begin{align}
  &\{ a^nb^na^n \mid n \geq 0 \} \cup 
  \{a^kb^lc^m \mid k \geq 0, l \geq 0, m \geq 20 \} =\nonumber\\
  =\: &\{ a^nb^na^n \mid 0 \leq n < 20 \} \cup
  \{ a^nb^na^n \mid n \geq 20 \} \cup
  \{a^kb^lc^m \mid k \geq 0, l \geq 0, m \geq 20 \} =\nonumber\\
  =\: &\{ a^nb^na^n \mid 0 \leq n \leq 20 \} \cup
  \{a^kb^lc^m \mid k \geq 0, l \geq 0, m \geq 20 \}
\end{align}

так как
\begin{align*}
  \{ a^nb^na^n \mid n \geq 20 \} \subset
  \{a^kb^lc^m \mid k \geq 0, l \geq 0, m \geq 20 \}
\end{align*}

Первое множество в (1) конечно, значит это регулярный язык. 
Второе множество описывается регулярным выражением $a^*b^*a^{20}a^*$,
значит этот язык тоже регулярный. Объеденение регулярных языков -
регулярный язык. Регулярные языки контекстно свободны.
    
    \end{document}
    