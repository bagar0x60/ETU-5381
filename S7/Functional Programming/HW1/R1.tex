\documentclass[a4paper,12pt]{article}
\usepackage[utf8]{inputenc}         % кодировка исходного текста
\usepackage[russian]{babel} % локализация и переносы

\usepackage{amsmath, amssymb, amsthm, graphicx, epsfig, fancyhdr}
\usepackage{cmap}                   % поиск в PDF


%%% Дополнительная работа с математикой
\usepackage{amsmath,amsfonts,amssymb,amsthm,mathtools} % AMS
\usepackage{icomma} % "Умная" запятая: $0,2$ --- число, $0, 2$ --- перечисление

%% Номера формул
\mathtoolsset{showonlyrefs=true} % Показывать номера только у тех формул, на которые есть \eqref{} в тексте.

%% Шрифты
\usepackage{euscript} 
\usepackage{mathrsfs} 
\usepackage{dsfont}

% для сноски в виде символа, а не номера
\usepackage[symbol]{footmisc}
\renewcommand{\thefootnote}{\fnsymbol{footnote}}

\newcounter{zcounter}

\newcommand{\z}{\par\addtocounter{zcounter}{1}%
\textsc{\fbox{\textbf{FP\arabic{zcounter}}}\quad} }

\newcommand{\zopt}{\par\addtocounter{zcounter}{1}%
\textsc{\fbox{FP\arabic{zcounter}}\quad} }

\DeclareMathOperator{\FV}{FV}
\DeclareMathOperator{\BV}{BV}

\textwidth=175mm
\oddsidemargin=-8mm
\topmargin=-8mm

\title{Функциональное программирование \\ Задание 1}
\author{ Кобылянский А.В. \\ Группа 5381 }
\date {}

\begin{document}

\pagenumbering{gobble}
\maketitle
\newpage
\pagenumbering{arabic}

\z Определите свободные и связанные переменные в термах и выполните подстановки:
\break
$\bullet \: [ux/x] \: (x\lambda y.yx)$
\begin{align*}
&\BV(x\lambda y.yx) = \{y\} \\
&\FV(ux) = \{u, x\}
\end{align*}
\begin{align*}
[ux/x] \: (x\lambda y.yx)  \equiv ux\lambda y.y(ux)
\end{align*}

$\bullet \: [uy/x] \: (x\lambda y.yx) $
\begin{align*}
&\BV(x\lambda y.yx) = \{y\} \\
&\FV(uy) = \{u, y\}
\end{align*}
\begin{align*}
[uy/x] \: (x\lambda y.yx)  \equiv_{\alpha} 
[uy/x] \: (x\lambda y'.y'x) \equiv
uy \lambda y'.y'(uy)
\end{align*}

$\bullet \: [xz/y] \: (xy(\lambda xz.xyz)y)$
\begin{align*}
&\BV(xy(\lambda xz.xyz)y) = \{x, z\} \\
&\FV(xz) = \{x, z\}
\end{align*}
\begin{align*}
[xz/y] \: (xy(\lambda xz.xyz)y) \equiv_\alpha
[xz/y] \: (xy(\lambda x'z'.x'yz')y) \equiv
x(xz)(\lambda x'z'.x'(xz)z')(xz)
\end{align*}

$\bullet \: [x(\lambda t. tt)s/x] \: (\lambda t. x (\lambda s.sx)t) $
\begin{align*}
&\BV(\lambda t. x (\lambda s.sx)t) = \{t, s\} \\
&\FV(x(\lambda t. tt)s) = \{x, s\}
\end{align*}
\begin{align*}
[x(\lambda t. tt)s/x] \: (\lambda t. x (\lambda s.sx)t) &\equiv_\alpha
[x(\lambda t. tt)s/x] \: (\lambda t. x (\lambda s'.s'x)t) \\
&\equiv \lambda t. x(\lambda t. tt)s (\lambda s'.s'(x(\lambda t. tt)s))t
\end{align*}

$\bullet \: [y(\lambda v. vx)/w]  \: ((\lambda x.x(\lambda y.yx)w)(\lambda w.wv)) $
\begin{align*}
&\BV((\lambda x.x(\lambda y.yx)w)(\lambda w.wv)) = \{x, y, w\} \\
&\FV(y(\lambda v. vx)) = \{y, x\}
\end{align*}
\begin{align*}
[y(\lambda v. vx)/w]  \: ((\lambda x.x(\lambda y.yx)w)(\lambda w.wv)) &\equiv_\alpha
[y(\lambda v. vx)/w]  \: ((\lambda x'.x'(\lambda y'.y'x')w)(\lambda w.wv)) \\
&\equiv (\lambda x'.x'(\lambda y'.y'x')(y\lambda v. vx)) \lambda w.wv
\end{align*}
\bigbreak

\z Уберите лишние скобки и при возможности выполните $\beta$ "---преобразование:
\break

$\bullet \:  ((\lambda x.(\lambda y.((xy)z))) (a(bc)) ) $ 
\begin{align*}
    ((\lambda x.(\lambda y.((xy)z))) (a(bc)) ) &\rightarrow 
    (\lambda xy.xyz) (a(bc)) \\ &\equiv_\beta
    \lambda y.a(bc)yz
\end{align*}

$\bullet \:  ((\lambda x.(\lambda y.((yx)x))) (x(xy))y)  $ 
\begin{align*}
    ((\lambda x.(\lambda y.((yx)x))) (x(xy))y) &\rightarrow 
    (\lambda xy.yxx) (x(xy)) y \\ &\equiv_\alpha
    (\lambda x'y'.y'x'x') (x(xy)) y \\ &\equiv_\beta
    y(x(xy))(x(xy))  
\end{align*}
\bigbreak

\z Покажите, что:
\[
\mathbf{SKK} = \mathbf{I}
\]
\begin{align*}
    \mathbf{SKK} &\equiv 
    (\lambda xyz.xz(yz))\mathbf{KK} \\ &\equiv_\beta
    (\lambda yz.\mathbf{K}z(yz))\mathbf{K} \\ &\equiv_\beta
    (\lambda yz.z)\mathbf{K} \\ &\equiv_\beta
    \lambda z.z \\ &\equiv_\alpha
    \mathbf{I} 
\end{align*}
\bigbreak

\z Определите терм для оператора $\textsc{xor}$.
\begin{align*}
    \mathop{XOR} &\equiv \lambda ab. a (\mathop{NOT}b) b     
\end{align*}
\bigbreak

\z Реализуйте для чисел Чёрча функцию возведения в степень.
\begin{align*}
    \mathds{1}  &\equiv \lambda s. \lambda z. sz\\
    \mathop{MULT} &\equiv \lambda a. \lambda b. \lambda s. \lambda z. a(bs)z\\   
    \mathop{POW} &\equiv \lambda a. \lambda b. b \: (\mathop{MULT} \: a) \: \mathds{1}
\end{align*}
$\mathop{POW}ab = a^b$ 
\bigbreak

\zopt Покажите, что:
\[
\mathbf{B} = \mathbf{S (K S) K}
\]
\begin{align*}
    \mathbf{S (K S) K} &\equiv (\lambda xyz.xz(yz))(\mathbf{KS})\mathbf{K} \\ &\equiv_\beta
    \lambda z. (\mathbf{KS})z(\mathbf{K}z) \\ &\equiv_\beta
    \lambda z. \mathbf{S} (\mathbf{K}z) \\ &\equiv
    \lambda z. (\lambda abc.ac(bc)) (\mathbf{K}z)  \\ &\equiv_\beta
    \lambda zbc.(\mathbf{K}z)c(bc)  \\ &\equiv_\beta
    \lambda zbc.z(bc) \\& \equiv
    \mathbf{B}
\end{align*}
\bigbreak

\zopt Мы определяли оператор \textsc{not} как:
\[
not \equiv \lambda b. b \: fls \: tru
\]
Найдите более <<короткую>> (в смысле длины терма см.~ниже) версию оператора \textsc{not}.\\
Длина терма $M$ ($lgh(M)$):
\[
\begin{aligned}
 & lgh(x)= 1, \:  x \in V \\ 
 & lgh(M N) = lgh(M) + lgh(N), \: M \in \Lambda \wedge N \in \Lambda \\
 & lgh(\lambda x.M ) = 1 + lgh(M), \: x \in V \wedge M \in \Lambda  \\
\end{aligned}
\]
Для $M \equiv \lambda b. b \: fls \: tru \equiv \lambda b. b \: (\lambda tf.f) \: (\lambda tf.t) $ длина терма равна $8$.
\begin{align*}
    \mathop{NOT} &\equiv \lambda b. \lambda t. \lambda f. bft \\
    lgh(\mathop{NOT}) &= 6
\end{align*}
\bigbreak

\zopt Реализуйте для чисел Чёрча функцию \textit{isEven}, возвращающую \textit{tru} на четных числах и \textit{fls} на нечетных.
\begin{align*}
    \mathop{isEven} &\equiv \lambda n. n \: \mathop{NOT} \: tru
\end{align*}

\end{document}